\documentclass[main.tex]{subfiles}

\begin{document}
\section{Summary}
The purpose of this work was to study and refine the \disp method, as well as to study the explicit and implicit dependencies of the reconstruction using this method. This revealed that:

\begin{itemize}
\item The \disp method is significantly better than the standard method of reconstruction at $\phi\gtrsim45^\circ$.
\item The angular 68\% containment for the new \disp tables is at the level of $0.3^\circ$ or better for energies above 1 TeV and zenith angles greater than $30^\circ$, and at the level of $0.15^\circ$ for energies above $\sim$3 TeV and zenith angles greater than $40^\circ$. At the largest zenith angles ($>55^\circ$), this constitutes {\bf improvements of 30-50\%} in angular resolution (see Fig. \ref{fig:energy_rel}).
\item The angular 68\% containment is independent of noise level in the training and testing sample.
\item The angular 68\% containment is independent of the camera acceptance correction (for offset from the center of the camera face) used in the training.
\item The angular 68\% containment is strongly dependent on energy with the expected trend - improving with both energy and zenith angle up to constraints placed by low statistics.
\item While the new \disp tables appear to perform worse than the older Disp tables in the small and medium zenith angle ranges (as measured by the 68\% containment), \textbf{the improvements in the LZA ranges are significant}. The performance in the small-medium zenith angle range is not a major concern since the standard method performs as well or better in these energy ranges and requires fewer resources.
\item The results obtained from an analysis of known objects are largely within 15\% of the those expected from the simulations. The angular reconstruction resolution determined from the Monte Carlo simulations, when tested using known source objects (Crab, Mrk 421), shows that (for $\phi\geq50$ and $E\geq 1$ TeV) Mrk 421 has a 68\% containment of $\sim 0.11^\circ$ and that (for $\phi\geq50$ and $E\geq 1$ TeV) the Crab has a 68\% containment of $0.15^\circ$.
\end{itemize}

\section{Future Work}

The data reconstruction deviates from the simulation reconstruction significantly, based on the currently available observed dataset. Part of this may be a question of small number statistics, and this could be probed by using small-number sub-samples of the simulation data-set and looking at the $\xi$ for the data. Alternatively, there may be a systematic energy-related mis-characterization of photons as ``on-source'' in the data leading to a bias towards the tails. Other explanations could be in the deviation from Gaussianity in the simulations themselves, which themselves merit a closer look.


\end{document}
%%% Local Variables:
%%% mode: latex
%%% TeX-master: "main"
%%% End:
