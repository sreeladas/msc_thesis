\documentclass[main.tex]{subfiles}

\begin{document}
La reconstruction g\'eom\'etrique de direction de cascades de particules au système d'imagerie t\'elescopique Cherenkov (IACT) de VERITAS souffre d'une d\'egradation substantielle aux grands angles z\'enithiques ($ \phi> 45 ^{\circ} $). Une reconstruction bas\'ee sur l'apprentissage machine ne devrait pas souffrir des m\^emes limitations car elle ne compte pas sur la g\'eom\'etrie de l'observation. Dans ce travail, nous d\'emontrons l'efficacit\'e pr\'evue d'une reconstruction \`a l'aide de Boosted Decision Trees optimis\'es. Nous testons \'egalement dans quelle mesure cela se traduit par une analyse de donn\'ees en effectuant une validation de principe avec des objets compacts et \`a source ponctuelle.

\end{document}
