\documentclass[main.tex]{subfiles}

\begin{document}
\section{Summary}
This purpose of this work was to study and refine the \disp method, as well as to study the explicit and implicit dependencies of the reconstruction using this method. This revealed that:

\begin{itemize}
\item The 68\% containment for the new \disp tables is at the level of $0.075^\circ$ or better for energies above 1 TeV and zenith angles greater than $30^\circ$, and at the level of $0.05^\circ$ for energies above 1TeV and zenith angles greater than $40^\circ$. At the largest zenith angles ($>55^\circ$), this constitutes {\bf improvements of 50-75\%}.
\item The 68\% containment is independent of noise level in the training and testing sample.
\item The 68\% containment is independent of the camera acceptance correction (for offset from the center of the camera face) used in the training.
\item The 68\% containment is dependent on energy with the expected trend - improving with both energy and zenith angle up to constraints placed by low statistics.
\item The reconstruction resolution determined from the Monte Carlo simulations, when tested using known source objects (Crab, Mrk 421), shows that Mrk 421 has a 68\% containment of $\sim 0.05^\circ$ and that the Crab has a 68\% containment of $0.05^\circ$.
\item While the new \disp tables appear to perform worse than the older Disp tables in the SZA and MZA ranges (as measured by the 68\% containment), the improvements in the LZA ranges are significant. The performance in the SZA-MZA range is not a major concern since the standard method performs as well or better in these energy ranges and requires fewer resources.
\item The results obtained from an analysis of known objects do not agree with those expected from the simulations.
\end{itemize}

\section{Discussion}
The results obtained from an analysis of known objects do not agree with those expected from the simulations.

\section{Future Work}


\end{document}