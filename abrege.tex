\documentclass[main.tex]{subfiles}

\begin{document}
La reconstruction g\'eom\'etrique de la direction de la cascade de particules dans le système de t\'elescope Cherenkov atmosph\'erique (IACT) d'imagerie VERITAS souffre d'une d\'egradation substantielle aux grands angles z\'enithiques ($ \phi> 45 ^{\circ} $). Une reconstruction bas\'e sur l'apprentissage automatique de cette direction ne devrait pas souffrir des m\^emes limitations car elle ne compte pas sur la g\'eom\'etrie de l'observation. Dans ce travail, nous d\'emontrons l'efficacit\'e pr\'evue d'une reconstruction \`a l'aide de Boosted Decision Trees optimis\'es. Nous testons \'egalement dans quelle mesure cela se traduit par une analyse de donn\'ees en effectuant une validation de principe avec des objets compacts et \`a source ponctuelle.

\end{document}
