\documentclass[main.tex]{subfiles}

\begin{document}
\section{Summary}
This purpose of this work was to recreate the scripts with which to create the BDT weight tables for the \disp method, as well as to study the explicit and implicit dependencies of the reconstruction using this method. This revealed that:

\begin{itemize}
\item The new \disp tables appear to perform worse than the older Disp tables in the SZA and MZA ranges (as measured by the 68\% containment). This is not seen as a major concern since the standard method performs as well or better in these energy ranges and requires fewer resources.
\item The 68\% containment for the new \disp tables is at the level of $0.07^\circ$ or better for energies above 1 TeV and zenith angles greater than $30^\circ$, and at the level of $0.03^\circ$ for energies above 1TeV and zenith angles greater than $40^\circ$.
\item The 68\% containment is independent of noise level in the training and testing sample.
\item The 68\% containment is independent of the acceptance correction (for offset from the center of the camera face) used in the training.
\item The 68\% containment is dependent on energy with the expected trend (better angular resolution at higher energy).
\item The reconstruction resolution determined from the Monte Carlo simulations, when tested using known source objects (Crab, Mrk 421), shows that Mrk 421 has a 68\% containment of [insert result] and that the Crab has a 68\% containment of [insert result].

\end{itemize}

\section{Discussion}
well

\section{Future Work}


\end{document}